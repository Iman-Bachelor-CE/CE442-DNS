\section*{جواب سوال ۳}

\subsection{ترافیک \lr{HTTP}}

برای این بخش، در ابتدا این فیلتر را در وایرشارک اعمال می‌کنیم:

\lr{http \&\& ip.src == 192.168.0.100}

خروجی وایرشارک بدین شکل می‌شود:



حالا ۳ تا از دامنه‌های بدست آمده این موارد هستند:

\begin{figure}[H]
	\centering
	\includegraphics[width=0.7\textwidth]{pic1.jpg}
	\caption{خروجی Wireshark}
	\label{fig:label4}
\end{figure}


\begin{figure}[H]
	\centering
	\includegraphics[width=0.7\textwidth]{pic2.jpg}
	\caption{دامنه: \lr{www.amazon.com} , آیپی: \lr{205.251.242.54}}
	\label{fig:label4}
\end{figure}

\begin{figure}[H]
	\centering
	\includegraphics[width=0.7\textwidth]{pic3.jpg}
	\caption{دامنه: \lr{z-ecx.images-amazon.com} , آیپی: \lr{23.3.96.123}}
	\label{fig:label4}
\end{figure}

\begin{figure}[H]
	\centering
	\includegraphics[width=0.7\textwidth]{pic4.jpg}
	\caption{دامنه: \lr{ad.doubleclick.net} , آیپی: \lr{74.125.225.91}}
	\label{fig:label4}
\end{figure}

\subsection{ترافیک \lr{FTP}}