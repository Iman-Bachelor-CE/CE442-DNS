\section*{جواب سوال ۳}

یکی از ویژگی‌های برجستهٔ nftables در مقایسه با iptables ، اتمیک بودن تغییرات در قوانین دیواره آتش است. منظور از اتمیک بودن این است که اعمال تغییرات در قوانین، به صورت یکپارچه و به طور کامل انجام می‌شود. به عبارت دیگر، در صورتی که عملیات تغییرات به هر دلیلی متوقف شود یا با شکست مواجه گردد، سیستم به حالت قبلی بازمی‌گردد و هیچ تغییری اعمال نمی‌شود.

این ویژگی از منظر امنیتی و پایداری شبکه بسیار اهمیت دارد، زیرا از ایجاد وضعیت‌های ناخواسته و آسیب‌پذیری در زمان به‌روزرسانی قوانین جلوگیری می‌کند.

در iptables ، قوانین به صورت جداگانه و به ترتیب اعمال می‌شوند. اگر در فرآیند اعمال یک سری از قوانین، خطا یا وقفه‌ای ایجاد شود، ممکن است برخی از قوانین اعمال شده و برخی دیگر اعمال نشده باشند. این وضعیت باعث ایجاد ناهماهنگی و آسیب‌پذیری در سیستم می‌شود.

اما در nftables ، تمامی تغییرات پیشنهادی ابتدا به صورت یک مجموعه در حافظه آماده می‌شوند و سپس در یک مرحله به طور کامل و یکجا بر روی سیستم اعمال می‌شوند. اگر به هر دلیلی این فرآیند با خطا مواجه شود، هیچ تغییری در قوانین فعلی دیواره آتش رخ نمی‌دهد و سیستم به حالت پایدار قبلی باقی می‌ماند.

\begin{figure}[H]
	\centering
	\includegraphics[width=0.7\textwidth]{pic17.jpg}
	\caption{سرور آماده‌ی دریافت}
	\label{fig:label4}
\end{figure}