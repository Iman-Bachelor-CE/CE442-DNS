\section*{جواب سوال ۳}

یکی از ویژگی‌های برجستهٔ nftables در مقایسه با iptables ، اتمیک بودن تغییرات در قوانین دیواره آتش است. منظور از اتمیک بودن این است که اعمال تغییرات در قوانین، به صورت یکپارچه و به طور کامل انجام می‌شود. به عبارت دیگر، در صورتی که عملیات تغییرات به هر دلیلی متوقف شود یا با شکست مواجه گردد، سیستم به حالت قبلی بازمی‌گردد و هیچ تغییری اعمال نمی‌شود.

این ویژگی از منظر امنیتی و پایداری شبکه بسیار اهمیت دارد، زیرا از ایجاد وضعیت‌های ناخواسته و آسیب‌پذیری در زمان به‌روزرسانی قوانین جلوگیری می‌کند.

در iptables ، قوانین به صورت جداگانه و به ترتیب اعمال می‌شوند. اگر در فرآیند اعمال یک سری از قوانین، خطا یا وقفه‌ای ایجاد شود، ممکن است برخی از قوانین اعمال شده و برخی دیگر اعمال نشده باشند. این وضعیت باعث ایجاد ناهماهنگی و آسیب‌پذیری در سیستم می‌شود.

اما در nftables ، تمامی تغییرات پیشنهادی ابتدا به صورت یک مجموعه در حافظه آماده می‌شوند و سپس در یک مرحله به طور کامل و یکجا بر روی سیستم اعمال می‌شوند. اگر به هر دلیلی این فرآیند با خطا مواجه شود، هیچ تغییری در قوانین فعلی دیواره آتش رخ نمی‌دهد و سیستم به حالت پایدار قبلی باقی می‌ماند.

جالا در این بخش این کامند را ابتدا ران می‌کنیم:

\lr{nc -l -k 127.0.0.1 8001}

سرور آماده‌ی دریافت می‌شود.

\begin{figure}[H]
	\centering
	\includegraphics[width=0.7\textwidth]{pic17.jpg}
	\caption{سرور آماده‌ی دریافت}
	\label{fig:label4}
\end{figure}

حالا در ادامه فایل
\lr{rules.nft}
را ایجاد می‌کنیم و قوانین مربوطه را در آن می‌نویسیم.

\begin{figure}[H]
	\centering
	\includegraphics[width=0.7\textwidth]{pic18.jpg}
	\caption{فایل \lr{rules.nft}}
	\label{fig:label4}
\end{figure}

در این اسکریپت، یک جدول برای فیلتر کردن ترافیک در محیط شبکه تعریف شده است. جدول با نام \texttt{filter} و در دامنه \texttt{inet} تعریف شده و شامل سه زنجیره می‌شود:

\begin{itemize}
	\item \textbf{زنجیره ورودی (input):}
	\begin{itemize}
		\item ترافیک ورودی بر اساس چندین قاعده مورد پذیرش قرار می‌گیرد.
		\item ترافیک از رابط \texttt{lo} (لوکال) بدون شرط پذیرفته می‌شود.
		\item ترافیکی که حالت آن \texttt{related} یا \texttt{established} است نیز پذیرفته می‌شود.
		\item ترافیک TCP با پورت‌های مقصد 8000 و 8001 پذیرفته می‌شود.
	\end{itemize}
	
	\item \textbf{زنجیره خروجی (output):}
	\begin{itemize}
		\item تمام ترافیک خروجی به طور پیش‌فرض پذیرفته می‌شود.
	\end{itemize}
	
	\item \textbf{زنجیره فوروارد (forward):}
	\begin{itemize}
		\item تمام ترافیک فوروارد به طور پیش‌فرض پذیرفته می‌شود.
	\end{itemize}
\end{itemize}

این قوانین برای ساده‌سازی مدیریت ترافیک شبکه و اطمینان از دسترسی بدون مانع به منابع محلی ضروری هستند.

حالا این فایل را به عنوان ورودی به دستور nft می‌دهیم:

\begin{figure}[H]
	\centering
	\includegraphics[width=0.7\textwidth]{pic19.jpg}
	\caption{ورودی دادن فایل}
	\label{fig:label4}
\end{figure}

حالا ارتباط برقرار است و پیام‌های مدنظرمان را وارد می‌کنیم.

\begin{figure}[H]
	\centering
	\includegraphics[width=0.7\textwidth]{pic20.jpg}
	\caption{تصویر سمت کارخواه}
	\label{fig:label4}
\end{figure}

\begin{figure}[H]
	\centering
	\includegraphics[width=0.7\textwidth]{pic21.jpg}
	\caption{تصویر سمت کارساز}
	\label{fig:label4}
\end{figure}