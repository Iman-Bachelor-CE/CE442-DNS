\section*{جواب سوال ۳}

یکی از ویژگی‌های برجستهٔ nftables در مقایسه با iptables ، اتمیک بودن تغییرات در قوانین دیواره آتش است. منظور از اتمیک بودن این است که اعمال تغییرات در قوانین، به صورت یکپارچه و به طور کامل انجام می‌شود. به عبارت دیگر، در صورتی که عملیات تغییرات به هر دلیلی متوقف شود یا با شکست مواجه گردد، سیستم به حالت قبلی بازمی‌گردد و هیچ تغییری اعمال نمی‌شود.

این ویژگی از منظر امنیتی و پایداری شبکه بسیار اهمیت دارد، زیرا از ایجاد وضعیت‌های ناخواسته و آسیب‌پذیری در زمان به‌روزرسانی قوانین جلوگیری می‌کند.

در iptables ، قوانین به صورت جداگانه و به ترتیب اعمال می‌شوند. اگر در فرآیند اعمال یک سری از قوانین، خطا یا وقفه‌ای ایجاد شود، ممکن است برخی از قوانین اعمال شده و برخی دیگر اعمال نشده باشند. این وضعیت باعث ایجاد ناهماهنگی و آسیب‌پذیری در سیستم می‌شود.

اما در nftables ، تمامی تغییرات پیشنهادی ابتدا به صورت یک مجموعه در حافظه آماده می‌شوند و سپس در یک مرحله به طور کامل و یکجا بر روی سیستم اعمال می‌شوند. اگر به هر دلیلی این فرآیند با خطا مواجه شود، هیچ تغییری در قوانین فعلی دیواره آتش رخ نمی‌دهد و سیستم به حالت پایدار قبلی باقی می‌ماند.

جالا در این بخش این کامند را ابتدا ران می‌کنیم:

\lr{nc -l -k 127.0.0.1 8001}

سرور آماده‌ی دریافت می‌شود.

\begin{figure}[H]
	\centering
	\includegraphics[width=0.7\textwidth]{pic17.jpg}
	\caption{سرور آماده‌ی دریافت}
	\label{fig:label4}
\end{figure}

حالا در ادامه فایل
\lr{rules.nft}
را ایجاد می‌کنیم و قوانین مربوطه را در آن می‌نویسیم.

\begin{figure}[H]
	\centering
	\includegraphics[width=0.7\textwidth]{pic18.jpg}
	\caption{فایل \lr{rules.nft}}
	\label{fig:label4}
\end{figure}

این کانفیگ شامل دو زنجیره در جدول \lr{NAT} است که برای دستکاری آدرس‌های IP استفاده می‌شود. این تغییرات شامل دو زنجیره \lr{prerouting} و \lr{output} است که هر دو برای اعمال تغییرات مشابهی بر روی پکت‌های IP هستند:

\begin{itemize}
	\item \textbf{زنجیره \lr{prerouting}}: این زنجیره قبل از روتینگ پکت‌ها اجرا می‌شود. به این ترتیب، تغییراتی که بر روی پکت‌ها اعمال می‌شود قبل از تعیین مسیر نهایی آن‌ها صورت می‌گیرد.
	
	\begin{itemize}
		\item \lr{type nat hook prerouting priority -100; policy accept;}:
		
		این خط مشخص می‌کند که این زنجیره از نوع \lr{NAT} بوده و در نقطه اتصال \lr{prerouting} با اولویت \lr{-100} قرار دارد. سیاست پیش‌فرض برای پکت‌هایی که مطابقت نمی‌یابند \lr{accept} است، به این معنی که پکت‌ها اجازه عبور دارند.
		
		\item \lr{ip daddr 127.0.0.1 tcp dport 8000 dnat to 127.0.0.1:8001}:
		
		این قانون می‌گوید پکت‌هایی که به آدرس IP \texttt{127.0.0.1} و پورت \texttt{8000} TCP می‌آیند، به \lr{127.0.0.1} در پورت \lr{8001} منتقل شوند (DNAT) .
	\end{itemize}
	
	\item \textbf{زنجیره \lr{output}}: این زنجیره پس از تصمیم‌گیری برای روتینگ و قبل از خروج پکت‌ها از ماشین اجرا می‌شود.
	
	\begin{itemize}
		\item \lr{type nat hook output priority -100; policy accept;}:
		
		مشابه با زنجیره \lr{prerouting}, این خط مشخص می‌کند که زنجیره در نقطه اتصال \lr{output} و با اولویت \lr{-100} فعال است. 
		
		\item \lr{ip daddr 127.0.0.1 tcp dport 8000 dnat to 127.0.0.1:8001}:
		
		همان تغییر آدرس و پورت که در \lr{prerouting} اعمال شده است، در اینجا نیز تکرار می‌شود.
	\end{itemize}
\end{itemize}

حالا این فایل را به عنوان ورودی به دستور nft می‌دهیم:

\begin{figure}[H]
	\centering
	\includegraphics[width=0.7\textwidth]{pic19.jpg}
	\caption{ورودی دادن فایل}
	\label{fig:label4}
\end{figure}

حالا ارتباط برقرار است و پیام‌های مدنظرمان را وارد می‌کنیم.

\begin{figure}[H]
	\centering
	\includegraphics[width=0.7\textwidth]{pic20.jpg}
	\caption{تصویر سمت کارخواه}
	\label{fig:label4}
\end{figure}

\begin{figure}[H]
	\centering
	\includegraphics[width=0.7\textwidth]{pic21.jpg}
	\caption{تصویر سمت کارساز}
	\label{fig:label4}
\end{figure}

حالا در ادامه قوانین را برای حالت UDP می‌نویسیم.

برای این قسمت باید برنامه netcat را به صورت همزمان بر روی درگاه‌های UDP ۸۰۰۱ و ۸۰۰۲ اجرا کنیم و دیواره آتش را طوری پیکربندی کنیم که بسته‌های UDP با مقصد درگاه‌های ۸۰۰۰ تا ۹۰۰۰ به طور یکنواخت به یکی از این دو درگاه ارسال شوند.

\begin{figure}[H]
	\centering
	\includegraphics[width=0.7\textwidth]{pic22.jpg}
	\caption{قوانین UDP}
	\label{fig:label4}
\end{figure}

\begin{figure}[H]
	\centering
	\includegraphics[width=0.7\textwidth]{pic23.jpg}
	\caption{پیام‌های ارسالی}
	\label{fig:label4}
\end{figure}

\begin{figure}[H]
	\centering
	\includegraphics[width=0.7\textwidth]{pic24.jpg}
	\caption{پیام‌های دریافتی در یک پورت}
	\label{fig:label4}
\end{figure}

\begin{figure}[H]
	\centering
	\includegraphics[width=0.7\textwidth]{pic25.jpg}
	\caption{پیام‌های دریافتی در یک پورت دیگر}
	\label{fig:label4}
\end{figure}

و اما تفاوت بین دیواره‌های آتش حالت‌دار و بی‌حالت،

دیواره آتش حالت‌دار \lr{(Stateful-Firewall)} : این نوع دیواره آتش با پیگیری وضعیت اتصال‌ها (مانند اتصال‌های جدید، موجود یا بسته‌شده) به کمک اطلاعات وضعیت (مانند \lr{conntrack} ) می‌تواند ترافیک را بر اساس ارتباطات جاری مدیریت کند. این توانایی باعث می‌شود تصمیم‌گیری درباره اجازه یا رد ترافیک دقیق‌تر باشد.

دیواره آتش بی‌حالت \lr{(Stateless-Firewall)} : این نوع دیواره آتش بسته‌ها را به صورت جداگانه و بدون توجه به وضعیت اتصال پردازش می‌کند. این مدل ساده‌تر است ولی برای سناریوهای پیچیده مدیریت ترافیک، مانند
\lr{Port Knocking}
یا ارتباطات وابسته به وضعیت، مناسب نیست.
 
حالا برای پیکربندی دیواره آتش با استفاده از nftables و پیاده‌سازی \lr{Port Knocking} برای باز کردن درگاه SSH ، ابتدا قوانین \lr{server_firewall} را می‌نویسیم و سپس بسته‌هایی به درگاه‌های ۱۳۳۷ و ۱۳۳۸ می‌فرستیم تا تست کنیم.
 
 قوانین نوشته شده به شرح زیر هستند:
 
 \begin{figure}[H]
 	\centering
 	\includegraphics[width=0.7\textwidth]{pic26.jpg}
 	\caption{قوانین \lr{server_firewall}}
 	\label{fig:label4}
 \end{figure}
 
\subsection*{توضیح قوانین}
 
\subsubsection*{جدول \lr{inet server_firewall}:}
 
 این جدول برای مدیریت فیلتر ترافیک استفاده می‌شود.
 
\subsubsection*{زنجیره \lr{input}:}
 
\begin{itemize}
	\item \lr{policy drop}: به صورت پیش‌فرض تمامی ترافیک‌های ورودی رد می‌شوند.
	\item \lr{iif lo accept}: ترافیک مربوط به \lr{loopback (lo)} مجاز است.
	\item \lr{ct state related, established accept}: بسته‌های مربوط به ارتباطات موجود یا ارتباطات قبلی مجاز هستند.
	\item \lr{tcp dport \{80, 443\} accept}: دسترسی به درگاه‌های 80 و 443 برای خدمات وب باز است.
	\item \lr{ct state invalid drop}: بسته‌هایی که وضعیت نامعتبر دارند رد می‌شوند.
	\item \lr{tcp dport 1337 mark set 0x1}: در صورت دریافت بسته به درگاه 1337، \lr{mark} مقدار \lr{0x1} می‌گیرد.
	\item \lr{tcp dport 1338 mark set 0x2 meta mark and 0x3 == 0x3 accept}: پس از دریافت بسته به درگاه 1338، اگر \lr{mark} قبلی مقدار \lr{0x1} داشته باشد (مطابق شرط \lr{meta mark and 0x3 == 0x3})، اجازه دسترسی به پورت داده می‌شود.
	\item \lr{tcp dport 22 meta mark and 0x3 == 0x3 accept}: دسترسی به پورت 22 (\lr{SSH}) تنها در صورت صحیح بودن \lr{mark} باز می‌شود.
	\item \lr{drop}: سایر ترافیک‌ها رد می‌شوند.
\end{itemize}
 
\subsubsection*{زنجیره \lr{forward}:}
 
\begin{itemize}
	\item تمامی ترافیک‌های عبوری رد می‌شوند.
\end{itemize}
 
\subsubsection*{زنجیره \lr{output}:}
 
\begin{itemize}
	\item تمامی ترافیک‌های خروجی مجاز هستند.
\end{itemize}