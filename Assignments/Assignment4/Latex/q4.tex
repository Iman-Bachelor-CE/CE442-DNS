\section*{جواب سوال ۴}

\subsection*{ترافیک \lr{HTTP}}

برای این بخش، در ابتدا این فیلتر را در وایرشارک اعمال می‌کنیم:

\lr{http \&\& ip.src == 192.168.0.100}

خروجی وایرشارک بدین شکل می‌شود:

\begin{figure}[H]
	\centering
	\includegraphics[width=0.7\textwidth]{pic1.jpg}
	\caption{خروجی Wireshark}
	\label{fig:label4}
\end{figure}

حالا ۳ تا از دامنه‌های بدست آمده این موارد هستند:

\begin{figure}[H]
	\centering
	\includegraphics[width=0.7\textwidth]{pic2.jpg}
	\caption{دامنه: \lr{www.amazon.com} , آیپی: \lr{205.251.242.54}}
	\label{fig:label4}
\end{figure}

\begin{figure}[H]
	\centering
	\includegraphics[width=0.7\textwidth]{pic3.jpg}
	\caption{دامنه: \lr{ad.adlegend.com} , آیپی: \lr{74.217.34.2}}
	\label{fig:label4}
\end{figure}

\begin{figure}[H]
	\centering
	\includegraphics[width=0.7\textwidth]{pic4.jpg}
	\caption{دامنه: \lr{ad.doubleclick.net} , آیپی: \lr{74.125.225.91}}
	\label{fig:label4}
\end{figure}

حالا در این بخش از ۲ کوئری استفاده می‌کنیم.


\begin{figure}[H]
	\centering
	\includegraphics[width=0.7\textwidth]{pic5.jpg}
	\caption{خروجی کوئری \lr{http \&\& ip.src == 192.168.0.100 \&\& http.request.uri contains "/search?"}}
	\label{fig:label4}
\end{figure}

\begin{figure}[H]
	\centering
	\includegraphics[width=0.7\textwidth]{pic6.jpg}
	\caption{اطلاعات کوئری}
	\label{fig:label4}
\end{figure}

\begin{figure}[H]
	\centering
	\includegraphics[width=0.7\textwidth]{pic7.jpg}
	\caption{خروجی کوئری \lr{http \&\& ip.src == 192.168.0.100 \&\& http.request.uri contains "/search"}}
	\label{fig:label4}
\end{figure}

\begin{figure}[H]
	\centering
	\includegraphics[width=0.7\textwidth]{pic8.jpg}
	\caption{اطلاعات کوئری}
	\label{fig:label4}
\end{figure}

\subsection*{ترافیک \lr{FTP}}

برای این بخش، در ابتدا این فیلتر را در وایرشارک اعمال می‌کنیم:

\lr{ftp \&\& ip.src==192.168.0.100}

خروجی وایرشارک بدین شکل می‌شود:

\begin{figure}[H]
	\centering
	\includegraphics[width=0.7\textwidth]{pic9.jpg}
	\caption{خروجی Wireshark}
	\label{fig:label4}
\end{figure}

پس اطلاعات مدنظر ما بدین شکل است:

\begin{figure}[H]
	\centering
	\includegraphics[width=0.7\textwidth]{pic10.jpg}
	\caption{نام کاربری \lr{shiningmoon} و پسورد \lr{public}}
	\label{fig:label4}
\end{figure}

می‌دانیم که در پروتکل FTP برای دانلود یک فایل از سرور به کلاینت از کامند RETR استفاده می‌شود.

پس برای این بخش، این فیلتر را در وایرشارک اعمال می‌کنیم:

\lr{ftp \&\& ip.src==192.168.0.100 \&\& ftp.request.command == RETR}

خروجی وایرشارک بدین شکل می‌شود:

\begin{figure}[H]
	\centering
	\includegraphics[width=0.7\textwidth]{pic11.jpg}
	\caption{فایل‌های دانلود شده توسط کاربر از این سرور}
	\label{fig:label4}
\end{figure}

لیست فایل‌ها:

\lr{dragon.zip} ,
\lr{arp.java} ,
\lr{l2switch.java} ,
\lr{phase1.html}

\begin{figure}[H]
	\centering
	\includegraphics[width=0.7\textwidth]{pic12.jpg}
	\caption{لیستی از فایل‌ها که سمت سرور قرار دارند}
	\label{fig:label4}
\end{figure}

برای مثال ۲ فایلی که سمت سرور وجود دارد ولی کاربر دانلود نکرده:

\lr{jerrygen.zip} ,
\lr{mylog_Sat-May-05-17-27-06-CST-2012.txt}

\subsection*{ترافیک \lr{POP}}

برای این بخش، در ابتدا این فیلتر را در وایرشارک اعمال می‌کنیم:

\lr{pop}

خروجی وایرشارک بدین شکل می‌شود:

\begin{figure}[H]
	\centering
	\includegraphics[width=0.7\textwidth]{pic13.jpg}
	\caption{خروجی Wireshark}
	\label{fig:label4}
\end{figure}

پس اطلاعات مدنظر ما بدین شکل است:

\begin{figure}[H]
	\centering
	\includegraphics[width=0.7\textwidth]{pic14.jpg}
	\caption{نام کاربری \lr{cs155@dummymail.com} و پسورد \lr{whitehat}}
	\label{fig:label4}
\end{figure}

در بخش بعدی می‌بینیم که ۵ پیام در صندوق پستی کاربر وجود دارد.

\begin{figure}[H]
	\centering
	\includegraphics[width=0.7\textwidth]{pic15.jpg}
	\caption{۵ پیام}
	\label{fig:label4}
\end{figure}

محتویات یک ایمیل دل‌خواه نیز بدین شکل است:

\begin{figure}[H]
	\centering
	\includegraphics[width=0.7\textwidth]{pic16.jpg}
	\caption{ایمیل دل‌خواه}
	\label{fig:label4}
\end{figure}

تاریخ:
\lr{Fri, 23 Apr 2010 08:20:52 -0700}

از:
\lr{joe <cs155@dummymail.com>}

به:
\lr{cs155@dummymail.com}

موضوع:
\lr{foobar}

محتوا:
\lr{foobar}
