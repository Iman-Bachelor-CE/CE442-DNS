\section*{جواب سوال ۲}

\subsection*{بخش اول}

وقتی یک پیام
\lr{m}
ابتدا توسط الگوریتم فشرده‌سازی
\lr{C(m)}
فشرده می‌شود و سپس با استفاده از رمزنگاری بلوکی با اندازه بلوک ۸ بیت پد شده و رمزگذاری می‌گردد، طول نهایی پیام رمزگذاری شده مستقیماً به طول پیام فشرده شده
\lr{C(m)}
بستگی دارد. مهاجم که تنها تعداد بلوک‌های ارسال شده را مشاهده می‌کند، می‌تواند اطلاعاتی درباره طول
\lr{C(m)}
به دست آورد. از آنجایی که طول 
\lr{C(m)}
نشان‌دهنده تعداد و طول رشته‌های 1 متوالی در پیام اصلی
\lr{m}
است، مهاجم می‌تواند استنتاج‌هایی درباره الگوی تکراری در m انجام دهد.

با یک مثال درباره‌ی این مورد بیشتر توضیح می‌دهم. \textbf{مثال:}

پیام $m_{1}$: 1111111111011111010111

$C(m_{1}) = 1010010100010011$

طول $C(m_{1}) = 16$ بیت $\rightarrow$ ۲ بلوک ۸ بیتی

پیام $m_{2}$: 1111101111101111101111

$C(m_{2})$ ممکن است طول بیشتری داشته باشد (فرضاً 24 بیت) $\rightarrow$ ۳ بلوک ۸ بیتی

در این حالت، اگر مهاجم مشاهده کند که دو بلوک ارسال شده‌اند، احتمال می‌دهد پیام $m_{1}$ با الگوی تکراری مناسب فشرده شده است. اگر سه بلوک ارسال شود، ممکن است پیام $m_{2}$ با الگوی کمتر تکراری فشرده شده باشد.

\subsection*{بخش دوم}

مهاجم می‌تواند از تفاوت‌های فشرده‌سازی بین دو پیام با طول اصلی برابر استفاده کند تا تشخیص دهد کدام پیام رمزگذاری شده است. اگر دو پیام $m_{1}$ و $m_{2}$ دارای طول اصلی یکسان باشند اما الگوهای متفاوتی از رشته‌های 1 متوالی داشته باشند، فشرده‌سازی آن‌ها نیز متفاوت خواهد بود. پیام با الگوی فشرده‌تر (بیشتر رشته‌های تکراری و طولانی‌تر 1ها) منجر به $C(m)$ کوتاه‌تر و در نتیجه تعداد بلوک‌های کمتر خواهد شد.

با یک مثال درباره‌ی این مورد بیشتر توضیح می‌دهم. \textbf{مثال:}

پیام $m_{1}$: 1111111111011111010111

$C(m_{1}) = 1010010100010011$

تعداد بلوک‌های رمزگذاری شده: ۲ بلوک

پیام $m_{2}$: 1111101111101111101111

$C(m_{2}) = 1010101001010101$

تعداد بلوک‌های رمزگذاری شده: ۳ بلوک

در این حالت، اگر مهاجم مشاهده کند که تعداد بلوک‌های رمزگذاری شده ۲ بلوک است، نتیجه می‌گیرد که پیام $m_{1}$ رمزگذاری شده است. اگر ۳ بلوک مشاهده کند، نتیجه می‌گیرد که پیام $m_{2}$ رمزگذاری شده است.

این روش به مهاجم اجازه می‌دهد تا با تحلیل طول پیام‌های رمزگذاری شده و مقایسه آن‌ها با الگوهای مختلف فشرده‌سازی، اطلاعاتی درباره محتوای پیام اصلی بدست آورد. این نوع حمله نشان‌دهنده آسیب‌پذیری‌های احتمالی در ترکیب فشرده‌سازی و رمزنگاری است که پس از حملات 
\lr{CRIME}
 در پروتکل‌های جدید مانند
\lr{TLS 1.3}
حذف شده‌اند تا از چنین تهدیداتی جلوگیری شود.
