\section*{جواب سوال ۱}

\section*{پاسخ بخش آ:}

\subsection*{آ. مراحل تبادل کلید بین آزاده و بهرام در پروتکل \lr{X3DH} :}

در پروتکل X3DH که بخشی از پروتکل Signal است، تبادل کلید به صورت زیر انجام می‌شود:

\subsubsection*{کلیدهای بلندمدت:}

\begin{itemize}
	\item \textbf{آزاده (A) :} دارای یک جفت کلید بلندمدت ( IKA و SKA ) است که کلید عمومی IKA و کلید خصوصی SKA هستند.
	\item \textbf{بهرام (B) :} نیز دارای جفت کلید بلندمدت ( IKB و SKB ) می‌باشد.
\end{itemize}

\subsubsection*{کلیدهای پیش‌نویس (Prekeys) :}

\begin{itemize}
	\item \textbf{بهرام (B) :} علاوه بر کلیدهای بلندمدت، یک کلید 
	\lr{Signed PreKey (SPKB)}
	دارد که با کلید بلندمدت خود امضا شده است. همچنین مجموعه‌ای از کلیدهای یک‌بار مصرف 
	\lr{(OPKB)}
	را در سرور منتشر می‌کند.
\end{itemize}

\subsubsection*{فرآیند تبادل کلید:}

\paragraph{آزاده (A) :}

\begin{enumerate}
	\item ابتدا اطلاعات کلیدهای عمومی بهرام ( IKB ، SPKB و OPKB ) را از سرور دریافت می‌کند.
	\item یک کلید موقت (EKA) تولید می‌کند.
	\item محاسبات دیفی-هلمن را با استفاده از کلیدهای دریافتی انجام می‌دهد:
	\[
	DH1 = DH(IKA, SPKB)
	\]
	\[
	DH2 = DH(EKA, IKB)
	\]
	\[
	DH3 = DH(EKA, SPKB)
	\]
	\[
	DH4 = DH(EKA, OPKB)
	\]
	\item از یک تابع مشتق کلید (KDF) برای ترکیب نتایج دیفی-هلمن و تولید کلید مشترک (SK) استفاده می‌کند:
	\[
	SK = KDF(DH1 || DH2 || DH3 || DH4)
	\]
	\item پیام اولیه شامل IKA ، EKA و شناسه‌های OPKB مورد استفاده را به بهرام ارسال می‌کند.
\end{enumerate}

\paragraph{بهرام (B) :}

\begin{enumerate}
	\item پیام اولیه از آزاده را دریافت می‌کند.
	\item با استفاده از اطلاعات دریافتی و کلیدهای خود، همان محاسبات دیفی-هلمن را انجام داده و کلید مشترک SK را محاسبه می‌کند.
\end{enumerate}

\subsubsection*{نتیجه:}

هر دو طرف، آزاده و بهرام، به یک کلید مشترک SK دست می‌یابند که برای رمزنگاری ارتباطات بعدی استفاده می‌شود.

\section*{پاسخ بخش ب:}

\subsection*{ب. اجزای قابل تعویض و اجزای ثابت در پروتکل \lr{X3DH} :}

در پروتکل \lr{X3DH} ، برخی اجزا به طور منظم تعویض می‌شوند تا امنیت سیستم افزایش یابد، در حالی که برخی اجزا باید ثابت باقی بمانند تا هویت کاربران حفظ شود.

\subsubsection*{اجزای قابل تعویض:}

\begin{itemize}
	\item \textbf{کلیدهای PreKey یک‌بار مصرف (OPKB) :} این کلیدها تنها برای یک نشست خاص استفاده می‌شوند و پس از استفاده حذف می‌گردند. تعویض مکرر OPKB ها از بازپخش و استفاده مجدد توسط مهاجم جلوگیری می‌کند.
	\item \textbf{کلید Signed PreKey (SPKB) :} این کلید به طور دوره‌ای (مثلاً هر چند هفته یک بار) تعویض می‌شود تا از خطر فاش شدن یا سوء استفاده از کلیدهای قدیمی جلوگیری شود و امنیت ارتباطات آینده تضمین شود.
\end{itemize}

\subsubsection*{اجزای ثابت:}

\begin{itemize}
	\item \textbf{کلیدهای هویت بلندمدت ( IKA و IKB ) :} این کلیدها باید ثابت باقی بمانند تا کاربران بتوانند هویت یکدیگر را به طور معتبر اثبات کنند. تغییر این کلیدها می‌تواند فرآیند احراز هویت را مختل کند و اعتماد کاربران را تحت تاثیر قرار دهد.
\end{itemize}

\subsubsection*{چرا برخی اجزا باید ثابت باشند؟}

\begin{itemize}
	\item \textbf{حفظ هویت کاربران:} کلیدهای هویت بلندمدت برای تأیید هویت کاربران ضروری هستند. اگر این کلیدها تغییر کنند، دیگر امکان تأیید هویت کاربر به درستی وجود نخواهد داشت.
	\item \textbf{یکپارچگی امنیت:} تعویض دوره‌ای OPKB و SPKB باعث می‌شود که حتی اگر یک کلید به خطر بیفتد، تنها ارتباطات آینده تحت تاثیر قرار گیرند و ارتباطات گذشته همچنان امن باقی بمانند.
\end{itemize}

\section*{پاسخ بخش ج:}

\subsection*{ج. اجزایی که اصالت دوجانبه را تضمین می‌کنند و راهکارهای مقابله با حمله شخص میانی:}

\subsubsection*{تضمین اصالت دوجانبه:}

\begin{itemize}
	\item \textbf{کلیدهای هویت بلندمدت ( IKA و IKB ) :} این کلیدها برای امضای دیجیتال پیام‌ها استفاده می‌شوند. هر کاربر با استفاده از کلید خصوصی خود پیام‌ها را امضا کرده و طرف مقابل با استفاده از کلید عمومی طرف دیگر صحت امضا را بررسی می‌کند.
	\item \textbf{کلید Signed PreKey (SPKB) :} این کلید با کلید هویت بلندمدت امضا شده است، که اطمینان می‌دهد SPKB از طرف صاحب اصلی آن صادر شده و قابل اعتماد است.
\end{itemize}

\subsubsection*{راهکارهای مقابله با حمله شخص میانی (Man-in-the-Middle) :}

\begin{itemize}
	\item \textbf{بررسی دقیق امضاهای دیجیتال:} کاربران باید تمامی امضاهای دیجیتال را به دقت بررسی کنند تا از صحت و اعتبار آن‌ها اطمینان حاصل کنند.
	\item \textbf{استفاده از پروتکل‌های امن‌تر مانند TLS :} افزودن لایه‌های امنیتی اضافی می‌تواند از نفوذ مهاجم در مسیر ارتباط جلوگیری کند.
	\item \textbf{احراز هویت چندعاملی:} استفاده از روش‌های احراز هویت اضافی مانند رمزهای یک‌بار مصرف یا گواهی‌های دیجیتال می‌تواند امنیت ارتباط را افزایش دهد و از حملات شخص میانی جلوگیری کند.
\end{itemize}

\subsubsection*{اگر سرور معتمد نباشد:}

\begin{itemize}
	\item \textbf{پروتکل‌های احراز هویت قوی:} کاربران می‌توانند از پروتکل‌هایی که نیاز به تأییدیه‌های اضافی دارند استفاده کنند تا از صحت هویت طرف مقابل اطمینان حاصل کنند.
	\item \textbf{تبادل کلید به صورت مستقیم:} در صورت امکان، تبادل کلید به صورت مستقیم بین کاربران بدون استفاده از سرور می‌تواند از حملات شخص میانی جلوگیری کند.
\end{itemize}

\section*{پاسخ بخش د:}

\subsection*{د. اجزایی که ویژگی‌های \lr{Forward Secrecy} و \lr{Backward Secrecy} را تضمین می‌کنند و روش‌های انجام آن‌ها:}

\subsubsection*{\lr{Forward Secrecy} (محرمانگی پیشرو):}

\begin{itemize}
	\item \textbf{کلیدهای یک‌بار مصرف (OPKB) :} استفاده از OPKB ها که تنها برای یک نشست خاص استفاده می‌شوند، تضمین می‌کند که حتی اگر یک کلید جلسه فعلی فاش شود، ارتباطات قبلی همچنان امن باقی می‌مانند.
	\item \textbf{کلید موقت (EKA) :} تولید کلید موقت برای هر نشست جدید باعث می‌شود که هر نشست دارای کلید جدید و مستقل باشد، که از دسترسی به ارتباطات گذشته جلوگیری می‌کند.
\end{itemize}

\subsubsection*{\lr{Backward Secrecy} (محرمانگی پسرو):}

\begin{itemize}
	\item \textbf{تولید کلیدهای جدید برای هر نشست:} با تولید کلیدهای موقت جدید برای هر جلسه و حذف کلیدهای قدیمی پس از استفاده، ارتباطات آینده نیز از دسترسی مهاجمان محافظت می‌شود.
	\item \textbf{حذف کلیدهای قدیمی پس از استفاده:} با حذف کلیدهای یک‌بار مصرف پس از استفاده، مهاجم نمی‌تواند از کلیدهای قدیمی برای رمزگشایی ارتباطات جدید استفاده کند.
\end{itemize}

\subsubsection*{چگونگی تضمین این ویژگی‌ها:}

\begin{itemize}
	\item \textbf{ترکیب کلیدهای موقت و یک‌بار مصرف:} پروتکل \lr{X3DH} همراه با پروتکل \lr{Double Ratchet} با استفاده از ترکیب کلیدهای موقت و یک‌بار مصرف، اطمینان می‌دهد که هر نشست جدید دارای کلید جدید و مستقل است.
	\item \textbf{حذف کلیدها پس از استفاده:} با حذف کلیدهای یک‌بار مصرف و کلیدهای موقت پس از استفاده، پروتکل از ایجاد نقاط ضعف برای مهاجمان جلوگیری می‌کند و امنیت ارتباطات را تضمین می‌نماید.
\end{itemize}

\section*{پاسخ بخش ه:}

\subsection*{ه. مدیریت پیام‌های از دست رفته در پروتکل \lr{X3DH} و \lr{Double Ratchet} :}

در شرایطی که برخی پیام‌ها به مقصد نرسند، پروتکل‌های \lr{X3DH} و \lr{Double Ratchet} از روش‌های مختلفی برای مدیریت این وضعیت استفاده می‌کنند تا امنیت و تداوم ارتباطات حفظ شود:

\subsubsection*{استفاده از کلیدهای یک‌بار مصرف جدید (OPKB) :}

در صورت از دست رفتن پیام‌ها، کاربران می‌توانند از OPKB های جدید برای ادامه ارتباطات استفاده کنند. این امر تضمین می‌کند که پیام‌های بعدی امن باقی می‌مانند و پیام‌های از دست رفته تاثیری بر امنیت ارتباط ندارند.

\subsubsection*{پروتکل \lr{Double Ratchet} :}

این پروتکل با استفاده از فرآیندی دو مرحله‌ای برای تبادل کلید، اطمینان می‌دهد که حتی اگر برخی پیام‌ها از دست بروند، پیام‌های بعدی بدون مشکل ارسال و دریافت می‌شوند. همچنین، مکانیزم‌های همگام‌سازی و حفاظت در برابر ارسال مجدد پیام‌ها (Replay Protection) به حفظ امنیت ارتباطات کمک می‌کنند.

\subsubsection*{پیام‌های ناهمگام \lr{(Out-of-Order Messages)}:}

سیستم قادر است پیام‌های ناهمگام را شناسایی کرده و آن‌ها را به ترتیب صحیح مرتب‌سازی کند تا ارتباط به درستی ادامه یابد.

\subsubsection*{بازسازی ارتباط:}

در صورت از دست رفتن پیام‌های کلیدی یا پیام‌های اصلی، کاربران می‌توانند مجدداً کلیدهای جدید تولید کرده و ارتباط را بازسازی کنند تا از ادامه امنیت ارتباط اطمینان حاصل شود.

\subsubsection*{نتیجه‌گیری:}

پروتکل‌های \lr{X3DH} و \lr{Double Ratchet} با استفاده از کلیدهای یک‌بار مصرف و مکانیزم‌های همگام‌سازی، توانایی مدیریت پیام‌های از دست رفته را دارند و از تداوم امنیت و محرمانگی ارتباطات اطمینان می‌دهند.

\section*{پاسخ بخش و:}

\subsection*{و. اهمیت OPKB و نقش SPKB در امنیت پروتکل:}

\subsubsection*{اهمیت OPKB (\lr{One-Time PreKey}) :}

\begin{itemize}
	\item \textbf{افزایش امنیت:} \lr{OPKB}ها تنها برای یک نشست خاص استفاده می‌شوند و پس از استفاده حذف می‌گردند، که این امر از استفاده مجدد و بازپخش آن‌ها توسط مهاجم جلوگیری می‌کند.
	\item \textbf{مقاومت در برابر حملات Replay:} با استفاده از \lr{OPKB}های یک‌بار مصرف، ارسال مجدد پیام‌ها یا استفاده از کلیدهای قدیمی برای نفوذ به ارتباطات جدید غیرممکن می‌شود.
\end{itemize}

\subsubsection*{نقش SPKB (\lr{Signed PreKey}) در امنیت پروتکل:}

\begin{itemize}
	\item \textbf{احراز هویت:} \lr{SPKB}ها با استفاده از کلید هویت بلندمدت امضا شده‌اند، که این امر اطمینان می‌دهد \lr{SPKB}ها از طرف صاحب اصلی خود صادر شده و قابل اعتماد هستند.
	\item \textbf{پشتیبانی از \lr{Forward Secrecy}:} \lr{SPKB}ها به طور دوره‌ای تعویض می‌شوند و همراه با \lr{OPKB}ها، ویژگی‌های \lr{Forward Secrecy} را تضمین می‌کنند، زیرا کلیدهای جدید امنیت ارتباطات جدید را حفظ می‌کنند حتی اگر کلیدهای قبلی فاش شوند.
\end{itemize}

\subsubsection*{اهمیت تأمین OPKB و نقش SPKB در امنیت مکالمه:}

\begin{itemize}
	\item \textbf{تأمین OPKB :} تامین کافی از \lr{OPKB}ها اطمینان می‌دهد که همیشه یک کلید یک‌بار مصرف برای برقراری ارتباط جدید در دسترس باشد. کمبود \lr{OPKB}ها ممکن است منجر به استفاده مجدد از \lr{OPKB}های قبلی شود که امنیت ارتباطات را کاهش می‌دهد.
	\item \textbf{نقش SPKB :} \lr{SPKB}ها به عنوان کلیدهای امضا شده توسط کلید هویت بلندمدت، نقش مهمی در احراز هویت و تضمین اصالت ارتباطات دارند. این کلیدها مانع از ایجاد \lr{SPKB}های جعلی توسط مهاجم می‌شوند و امنیت ارتباطات را حفظ می‌کنند.
\end{itemize}

\subsubsection*{تأثیر کمبود OPKB و SPKB بر امنیت مکالمه:}

\begin{itemize}
	\item \textbf{کمبود OPKB :} ممکن است کاربران مجبور به استفاده مجدد از \lr{OPKB}های قبلی شوند که این امر می‌تواند امنیت ارتباطات را تهدید کند، زیرا کلیدهای تکراری ممکن است توسط مهاجم شناسایی و مورد سوء استفاده قرار گیرند.
	\item \textbf{کمبود SPKB :} نبود \lr{SPKB}های جدید می‌تواند احراز هویت کاربران را مختل کند و مهاجم ممکن است بتواند حملات جعل هویت را انجام دهد. همچنین، نبود \lr{SPKB}های جدید ویژگی‌های \lr{Forward Secrecy} را تحت تاثیر قرار می‌دهد و امنیت ارتباطات جدید را کاهش می‌دهد.
\end{itemize}

\section*{پاسخ بخش ز:}

\subsection*{ز. نیاز آزاده به انتظار آنلاین شدن بهرام برای تبادل کلید در صورت آفلاین بودن بهرام:}

خیر، آزاده نیازی ندارد که منتظر آنلاین شدن بهرام بماند تا تبادل کلید انجام شود. دلیل این امر استفاده از کلیدهای پیش‌تعیین شده و ذخیره شده توسط بهرام در سرور است.

\subsubsection*{توضیح:}

\begin{itemize}
	\item \textbf{کلیدهای \lr{PreKey} از قبل آپلود شده:} بهرام قبلاً مجموعه‌ای از کلیدهای یک‌بار مصرف (\lr{OPKB}) و کلیدهای \lr{Signed PreKey (SPKB)} را به سرور \lr{Signal} آپلود کرده است.
	\item \textbf{دریافت کلیدها توسط آزاده:} هنگامی که آزاده قصد ارسال پیام به بهرام را دارد، می‌تواند این کلیدهای یک‌بار مصرف را از سرور دریافت کند بدون نیاز به آنلاین بودن بهرام.
	\item \textbf{ارسال پیام اولیه:} آزاده با استفاده از این کلیدهای دریافت شده، پیام اولیه خود را به بهرام ارسال می‌کند. وقتی بهرام آنلاین می‌شود، پیام اولیه را دریافت و پردازش می‌کند و ارتباط رمزگذاری شده را برقرار می‌کند.
\end{itemize}

\subsubsection*{مزایا:}

\begin{itemize}
	\item \textbf{ایجاد ارتباط حتی در شرایط آفلاین:} این روش امکان ارسال پیام‌ها به کاربران آفلاین را فراهم می‌کند بدون اینکه ارسال‌کننده نیاز به انتظار داشته باشد.
	\item \textbf{افزایش انعطاف‌پذیری و کارایی:} کاربران می‌توانند در هر زمانی با ارسال پیام‌های امن ارتباط برقرار کنند بدون نیاز به هماهنگی زمان‌های آنلاین.
\end{itemize}

\section*{پاسخ بخش ح:}

\subsection*{ح. انکارپذیری در پروتکل سیگنال و تأثیر آن بر اصالت پیام‌ها:}

\subsubsection*{تعریف انکارپذیری:}

انکارپذیری به معنای این است که فرستنده پیام نمی‌تواند سپس ادعای نکند که آن پیام را ارسال کرده است. در پروتکل سیگنال، این ویژگی به گونه‌ای پیاده‌سازی شده که فرستنده نمی‌تواند بعداً از ارسال پیام انکار کند.

\subsubsection*{تضاد انکارپذیری با اطمینان از اصالت پیام‌ها:}

\begin{itemize}
	\item \textbf{اصالت پیام‌ها:} اصالت پیام‌ها تضمین می‌کند که پیام از طرف فرستنده واقعی ارسال شده است.
	\item \textbf{انکارپذیری:} در حالی که اصالت پیام‌ها را تضمین می‌کند، انکارپذیری باعث می‌شود که فرستنده نمی‌تواند پس از ارسال پیام آن را انکار کند.
\end{itemize}

\subsubsection*{جزئیات بیشتر:}

\paragraph{انکارپذیری توسط هر دو طرف:}

در پروتکل سیگنال، هر دو طرف مکالمه می‌توانند پیام‌ها را انکار کنند زیرا پیام‌ها با استفاده از کلیدهای موقت امضا می‌شوند که تنها طرف مقابل قادر به تأیید آن‌هاست. این امر باعث می‌شود که هیچ‌کدام از طرفین نتوانند بعداً ادعا کنند که پیام خاصی را ارسال نکرده‌اند.

\paragraph{ساخت مکالمات جعلی توسط اشخاص ثالث:}

فرد ثالثی مانند سودابه می‌تواند بدون داشتن هیچ پیام واقعی بین آزاده و بهرام، مجموعه‌ای از پیام‌های جعلی را ایجاد کند که به نظر می‌رسد از طرف یکی از کاربران ارسال شده‌اند. از آنجا که پیام‌های جعلی نیز با استفاده از کلیدهای یک‌بار مصرف و موقت ایجاد می‌شوند، کاربران نمی‌توانند به سادگی تشخیص دهند که این پیام‌ها واقعی هستند یا جعلی.

\subsubsection*{کاربردهای انکارپذیری در دنیای واقعی:}

\begin{itemize}
	\item \textbf{حفظ حریم خصوصی:} انکارپذیری به کاربران اجازه می‌دهد که پس از ارسال پیام‌ها، دیگر نتوانند آن‌ها را به‌طور قانونی برای اثبات ارسالشان مورد استفاده قرار دهند، که این امر حریم خصوصی را تقویت می‌کند.
	\item \textbf{پیشگیری از استفاده نادرست:} در مواردی که افراد ممکن است تحت فشار قانونی قرار گیرند تا پیام‌های خاصی را ارسال کنند، انکارپذیری به آن‌ها امکان می‌دهد که از الزام به ارسال پیام‌های ناخواسته جلوگیری کنند.
	\item \textbf{تسهیل ارتباطات آزاد و امن:} کاربران می‌توانند با اطمینان بیشتر و بدون نگرانی از ثبت دائمی پیام‌ها، ارتباطات خود را برقرار کنند.
\end{itemize}

\subsubsection*{نتیجه‌گیری:}

اگرچه انکارپذیری ممکن است در برخی موارد تضاد با تأیید اصالت پیام‌ها داشته باشد، اما در پروتکل سیگنال این دو ویژگی به گونه‌ای پیاده‌سازی شده‌اند که همزمان حفظ حریم خصوصی کاربران و امنیت ارتباطات امکان‌پذیر باشد. انکارپذیری در دنیای واقعی به کاربران کمک می‌کند تا ارتباطات خود را بدون ترس از سوء استفاده‌های قانونی یا ثبت دائمی پیام‌ها حفظ کنند.

در نتیجه پروتکل \lr{X3DH} و \lr{Double Ratchet} در پروتکل سیگنال با استفاده از ترکیب کلیدهای بلندمدت، موقت و یک‌بار مصرف، امنیت ارتباطات را از طریق ویژگی‌هایی مانند \lr{Forward Secrecy} و \lr{Backward Secrecy} تضمین می‌کنند. همچنین، با مدیریت دقیق تبادل کلید و جلوگیری از حملات شخص میانی، اصالت و امنیت ارتباطات حفظ می‌شود. اهمیت \lr{OPKB} و \lr{SPKB} در افزایش امنیت و جلوگیری از حملات \lr{Replay} و جعل هویت نقش کلیدی دارد. علاوه بر این، ویژگی انکارپذیری در پروتکل سیگنال با حفظ حریم خصوصی و جلوگیری از سوء استفاده‌های قانونی، به کاربران امکان می‌دهد تا ارتباطات خود را به صورت امن و مطمئن برقرار کنند.